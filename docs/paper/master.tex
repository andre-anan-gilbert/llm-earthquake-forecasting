%%%%%%%%%%%%%%%%%%%%%%%%%%%%%%%%%%%%%%%%%%%%%%%%%%%%%%%%%%
%   Autoren:
%   Prof. Dr. Bernhard Drabant
%   Prof. Dr. Dennis Pfisterer
%   Prof. Dr. Julian Reichwald
%%%%%%%%%%%%%%%%%%%%%%%%%%%%%%%%%%%%%%%%%%%%%%%%%%%%%%%%%%

%%%%%%%%%%%%%%%%%%%%%%%%%%%%%%%%%%%%%%%%%%%%%%%%%%%%%%%%%%
%	ANLEITUNG: 
%   1. Ersetzen Sie firmenlogo.jpg im Verzeichnis img
%   2. Passen Sie alle Stellen im Dokument an, die mit 
%      @stud 
%      markiert sind
%%%%%%%%%%%%%%%%%%%%%%%%%%%%%%%%%%%%%%%%%%%%%%%%%%%%%%%%%%

%%%%%%%%%%%%%%%%%%%%%%%%%%%%%%%%%%%%%%%%%%%%%%%%%%%%%%%%%%
%	ACHTUNG: 
%   Für das Erstellen des Literaturverzeichnisses wird das 
%   modernere Paket biblatex in Kombination mit biber 
%   verwendet - nicht mehr das ältere Paket BibTex!
%
%   Bitte stellen Sie Ihre TeX-Umgebung entsprechend ein (z.B. TeXStudio): 
%   Einstellungen --> Erzeugen --> Standard Bibliographieprogramm: biber
%%%%%%%%%%%%%%%%%%%%%%%%%%%%%%%%%%%%%%%%%%%%%%%%%%%%%%%%%%

\documentclass[fontsize=12pt,BCOR=5mm,DIV=12,parskip=half,listof=totoc,
               paper=a4,toc=bibliography,pointlessnumbers]{scrreprt}
               
\usepackage[utf8]{inputenc}

%% LANGUAGE SETTINGS
%
% @stud: Sprache ggf. anpassen
%
%\usepackage[ngerman]{babel} 	        % german language
%\usepackage[german=quotes]{csquotes} 	% correct quoting using \enquote{}
\usepackage[english]{babel}          % english language
\usepackage{csquotes} 	              % correct quoting using \enquote{}

%%%%%%%%%%%%%
%% ZITIERSTIL
%%%%%%%%%%%%%
%
% @stud: Zitierstil in package biblatex unten wählen
%
% NUMERIC Style - e. g. [12]
% style=numeric 
%
% IEEE Style - numeric kind of style 
% style=ieee 
%
% ALPHABETIC Style - e. g. [AB12]
% style=alphabetic 
%
% HARVARD Style 
% style=apa 
%
% CHICAGO Style 
% style=authoryear
%
% Position des Zitats:
%
% autocite=inline 
%
% (!!) FOOTNOTE POSITION NOT RECOMMENDED IN MINT DOMAIN:
% autocite=footnote
%
\usepackage[backend=biber, autocite=inline, style=apa]{biblatex} 	
\usepackage{makeidx}                  % allows index generation
\usepackage{listings}	                %Format Listings properly
\usepackage{lipsum}                   % Blindtext
\usepackage{graphicx}                 % use various graphics formats
\usepackage[german]{varioref}         % nicer references \vref
\usepackage{caption}	                % better Captions
\usepackage{booktabs}                 % nicer Tabs
\usepackage[hidelinks=true]{hyperref} % keine roten Markierungen bei Links
\usepackage{fnpct}                    % Correct superscripts 
\usepackage{calc}                     % Used for extra space below footsepline, in particular
\usepackage{array}
\usepackage{acronym}
\usepackage{algorithm}
\usepackage{algpseudocode}
\usepackage{setspace}
\usepackage{tocloft}
\usepackage[T1]{fontenc}
\usepackage{multirow}
\usepackage{amsmath}
\usepackage{amssymb}
\usepackage{rotating}

% Definitionen und Commands
\newcommand{\indextype}{numeric}
\newcommand{\abs}{\par\vskip 0.2cm\goodbreak\noindent}
\newcommand{\nl}{\par\noindent}
\newcommand{\mcl}[1]{\mathcal{#1}}
\newcommand{\nowrite}[1]{}
\newcommand{\NN}{{\mathbb N}}
\newcommand{\imagedir}{img}
\newcommand{\TitelDerArbeit}[1]{\def\DerTitelDerArbeit{#1}\hypersetup{pdftitle={#1}}}
\newcommand{\AutorDerArbeit}[1]{\def\DerAutorDerArbeit{#1}\hypersetup{pdfauthor={#1}}}
\newcommand{\Firma}[1]{\def\DerNameDerFirma{#1}}
\newcommand{\Kurs}[1]{\def\DieKursbezeichnung{#1}}
\newcommand{\Abteilung}[1]{\def\DerNameDerAbteilung{#1}}
\newcommand{\Studiengangsleiter}[1]{\def\DerStudiengangsleiter{#1}}
\newcommand{\WissBetreuer}[1]{\def\DerWissBetreuer{#1}}
\newcommand{\FirmenBetreuer}[1]{\def\DerFirmenBetreuer{#1}}
\newcommand{\Bearbeitungszeitraum}[1]{\def\DerBearbeitungszeitraum{#1}}
\newcommand{\Abgabedatum}[1]{\def\DasAbgabedatum{#1}}
\newcommand{\Matrikelnummer}[1]{\def\DieMatrikelnummer{#1}}
\newcommand{\Studienrichtung}[1]{\def\DieStudienrichtung{#1}}
\newcommand{\ArtDerArbeit}[1]{\def\DieArtDerArbeit{#1}}
\newcommand{\Literaturverzeichnis}{Literaturverzeichnis}

% Page Layout
\oddsidemargin=0mm
\evensidemargin=0mm
\textwidth=159mm
\topmargin=-18mm
\headsep=10mm
\textheight=251mm
\footheight=15mm

\makeindex

%%%%%%%%%%%%%%%%%%%%%%%%%%%%%%%%%%%
% LITERATURVERZEICHNIS
% @stud: Literaturverzeichnis in Datei bibliography.bib anpassen. 
%
% [Alternative zu Verwendung von \initializeBibliography: Citavi ... (dazu eigenes LaTex Coding verwenden)]
%
\addbibresource{bibliography.bib}
\DefineBibliographyStrings{ngerman}{andothers = {{et\,al\adddot}},}

% Elementare Konfigurationen und Definitionen werden geladen 
% @stud: gegebenenfalls anpassen
%
\input{config}

% @stud
%
% PERSÖNLICHE ANGABEN (BITTE VOLLSTÄNDIG EINGEBEN zwischen den Klammern: {...})
%
\ArtDerArbeit{Projekt} % "Bachelor" oder "Projekt" wählen
\TitelDerArbeit{Earthquake Forecasting}
\AutorDerArbeit{André Anan Gilbert, Felix Noll, Marc Grün}
\Kurs{WWI 21 DSB}
\Studienrichtung{Data Science}
\Matrikelnummer{3465546}
\Studiengangsleiter{Prof. Dr. Maximilian Scherer} % Kursleiter
\Bearbeitungszeitraum{06.05.2024 - 24.07.2024}
\Abgabedatum{dd.mm.yyyy}

\begin{document}

\setTitlepage

%%%%%%%%%%%%%%%%%%%%%%%%%%%%%%%%%%%
% EHRENWÖRTLICHE ERKLÄRUNG
%
% @stud: ewerkl.tex bearbeiten
%
% % !TEX root =  master.tex
\clearpage
\chapter*{Statutory Declaration}

% Wird die folgende Zeile auskommentiert, erscheint die ehrenwörtliche
% Erklärung im Inhaltsverzeichnis.

% \addcontentsline{toc}{chapter}{Ehrenwörtliche Erklärung}
I herewith declare that I have composed the present thesis with the title ``\textit{\DerTitelDerArbeit}'' myself and without use of any other than the cited sources and aids. Sentences or parts of sentences quoted literally are marked as such; other references with regard to the statement and scope are indicated by full details of the publications concerned.

The thesis in the same or similar form has not been submitted to any examination body and has not been published.

\vspace{3cm}
Place, Date \hfill \DerAutorDerArbeit
 
% \cleardoublepage  
%%%%%%%%%%%%%%%%%%%%%%%%%%%%%%%%%%%

%%%%%%%%%%%%%%%%%%%%%%%%%%%%%%%%%%%
% SPERRVERMERK
%
% @stud: nondisclosurenotice.tex bearbeiten
%
% % !TEX root =  master.tex
\chapter*{Non-Disclosure Notice}

The content of this work may not be made available to people outside of the examination and evaluation process, either as a whole or in part, unless otherwise authorized by the dual partner.

\cleardoublepage
 
% \cleardoublepage
%%%%%%%%%%%%%%%%%%%%%%%%%%%%%%%%%%%

%%%%%%%%%%%%%%%%%%%%%%%%%%%%%%%%%%%
%	KURZFASSUNG
%
% @stud: acknowledge.tex bearbeiten
%
% \input{acknowledge}
% \cleardoublepage 
%%%%%%%%%%%%%%%%%%%%%%%%%%%%%%%%%%%

%%%%%%%%%%%%%%%%%%%%%%%%%%%%%%%%%%%
% VERZEICHNISSE und ABSTRACT
%
% @stud: ggf. nicht benötigte Verzeichnisse auskommentieren/löschen
%
\tableofcontents
\cleardoublepage

% Abbildungsverzeichnis
\phantomsection
\addcontentsline{toc}{chapter}{\listfigurename}
\listoffigures
\cleardoublepage

%	Tabellenverzeichnis
% \phantomsection
% \addcontentsline{toc}{chapter}{\listtablename}
% \listoftables
% \cleardoublepage

%	Listingsverzeichnis / Quelltextverzeichnis
\lstlistoflistings
\cleardoublepage

% Algorithmenverzeichnis
% \listofalgorithms
% \cleardoublepage

% Abkürzungsverzeichnis
% @stud: acronyms.tex bearbeiten
% !TEX root =  master.tex
\clearpage
\chapter*{Acronyms}	
\addcontentsline{toc}{chapter}{Acronyms}

\begin{acronym}[XXXXXXX]
    \acro{ARIMA}{Autoregressive Integrated Moving Average}
    \acro{EMA}{Exponential Moving Average}
    \acro{FDSN}{International Federation of Digital Seismograph Networks}
    \acro{FFT}{Fast Fourier Transformation}
    \acro{LLM}{Large Language Model}
    \acro{LSTM}{Long Short-Term Memory}
    \acro{MAE}{Mean Absolute Error}
    \acro{MAPE}{Mean Absolute Percentage Error}
    \acro{MSE}{Mean Squared Error}
    \acro{IOT}{Internet of Things}
    \acro{PACF}{Partial Autocorrelation Functions}
    \acro{PCA}{Principal Component Analysis}
    \acro{RMSE}{Root Mean Squared Error}
    \acro{RNNs}{Recurrent Neural Networks}
    \acro{SMA}{Simple Moving Average}
    \acro{USGS}{United States Geological Survey}
\end{acronym} 
\cleardoublepage

\onehalfspacing

%	Kurzfassung / Abstract
% @stud: abstract.tex bearbeiten
\chapter*{Abstract}

Earthquakes are a common, yet sometimes catastrophic, phenomenon in our world. Approximately 2,000 earthquakes with a magnitude of 5 on the Richter scale occur annually \parencite{welt_erdbeben}. As this threat frequently endangers many lives, we aim to improve response time and enhance public understanding of earthquakes, particularly how to act in extreme situations. We have developed a CatBoost model to predict the magnitude and depth of an upcoming earthquake within the next three days for the 25 most vulnerable regions. Our model achieves a Mean Absolute Error (MAE) of 2.9344, a Root Mean Squared Error (RMSE) of 10.3577, and an R2-Score of 0.9495. Additionally, we have developed a \ac{LLM}-powered AI agent, enabling user interaction with the model results and providing answers to general questions about earthquakes to improve awareness and in-depth analysis of potential risks and safety measures. This project contributes to the field of forecasting research by exemplifying the use case of a categorical boosted tree model to forecast earthquakes.

\addcontentsline{toc}{chapter}{Abstract}
 
\cleardoublepage

\initializeText

%%%%%%%%%%%%%%%%%%%%%%%%%%%%%%%%%%%%%%%%%%%%%%%%%%%%%%%%%%%%%%%%%%%%%%%%%%%%%%%%%%%%%%%%%%
% KAPITEL UND ANHÄNGE
%
% @stud:
%   - nicht benötigte: auskommentieren/löschen
%   - neue: bei Bedarf hinzufügen mittels input-Kommando an entsprechender Stelle einfügen
%%%%%%%%%%%%%%%%%%%%%%%%%%%%%%%%%%%%%%%%%%%%%%%%%%%%%%%%%%%%%%%%%%%%%%%%%%%%%%%%%%%%%%%%%%

%%%%%%%%%%%%%%%%%%%%%%%%%%%%%%%%%%%
% KAPITEL
%
% @stud: einzelne Kapitel bearbeiten und eigene Kapitel hier einfügen
%
% Einleitung
% !TEX root =  master.tex

\chapter{Introduction}
Earthquakes are a common phenomenon in our world, with approximately 2,000 earthquakes
of magnitude 5 on the Richter scale occurring annually \parencite{welt_erdbeben}.
While these earthquakes are classified as moderate, they still have the potential
to be deadly and cause significant damage to infrastructure and communities. The
unpredictability of earthquakes poses a considerable challenge to disaster preparedness
and risk mitigation.

To mitigate the risks of injuries or death, our goal is to develop a forecasting model
capable of predicting when and where an earthquake might occur. Such a model would be
invaluable in providing early warnings, allowing for timely preparation and potentially
saving lives. By leveraging out-of-the-box forecasting models and adapting their methodology
to the use case of earthquake forecasting, we aim to build a machine learning model to
enhance the accuracy and reliability of earthquake forecasts.

To further expand our web platform, we plan to integrate an \ac{LLM}. This \ac{LLM}
will not only offer easy access to the calculated predictive information but will
also provide comprehensive insights into earthquakes. Our goal is to strengthen public
understanding of earthquakes and promote safety precautions. This additional resource
will serve as a tool, offering explanations on seismic activity, historical earthquake
data, and guidelines for preparedness. Thus, we aim not only to develop an approach
capable of predicting earthquakes but also to ensure long-term usage by the population
in endangered areas.

On one side, this page allows regions to stay ahead of potential earthquakes by offering
sufficient time for preparation or evacuation. Early warnings by our predictive model can
lead to better-organized emergency responses, reducing the impact of earthquakes on human
lives and property. By empowering communities with predictive data and educational resources,
we aim to enhance resilience against seismic hazards.

As a result, this paper is organized as follows: Chapter \ref{ch:forecasting} explains
forecasting and the theoretical background behind our approach in detail. Chapter
\ref{ch:application} describes the development and deployment of our model and web
application. Additionally, we evaluate the model's performance and feature importance.
Finally, we interpret the results and outline potential improvements and next steps in
Chapter \ref{ch:conclusion}.

% mehrere Grundlagen- und Forschungs-Kapitel
% !TEX root =  master.tex
\chapter{Time Series Forecasting}
\label{ch:forecasting}

Time series forecasting is a statistical technique used to predict future values based
on previously observed values. This method is particularly valuable in  fields, where
understanding and anticipating trends and patterns over time is crucial, such as
finance \parencite{andersen2005volatility}, medicine \parencite{topol2019high},
environmental science \parencite{mudelsee2019trend}, and biology \parencite{stoffer2012special}.

At its core, time series forecasting involves analyzing data points collected or
recorded at specific time intervals. These data points can represent anything from
daily stock prices and monthly sales figures to hourly weather readings and \ac{IOT}
sensor data. The primary objective is to use this historical data to make informed
predictions about future events, trends, or behaviors \parencite[ch. 1]{zhang2001investigation}.

One of the key characteristics of time series data is that the observations are
sequentially dependent. This temporal ordering introduces a unique challenge and
opportunity: the potential to identify patterns such as seasonality, trends, and
cycles \parencite{assfalg2009periodic}. For example, retail sales data often
exhibit seasonal patterns, with higher sales during holidays, while economic
indicators might show long-term trends reflecting economic growth or decline.

This chapter aims to give a detailed overview about important terminology and
time series forecasting models.

\section{Terminology}
To understand time series forecasting, we first need to define some terminology.
A time series is a set of observations recorded over time \parencite{haben2023time},
as shown in Figure \ref{tab:time_series_example}.


\begin{table}[h]
    \centering
    \begin{tabular}{|c|c|}
        \hline
        \textbf{Timestamp} & \textbf{Feature\_1} \\
        \hline
        2000-04-01         & 100                 \\
        \hline
        2000-04-02         & 150                 \\
        \hline
        2000-04-03         & 170                 \\
        \hline
    \end{tabular}
    \caption{A simple example for a time series.}
    \label{tab:time_series_example}
\end{table}

There are two unique features in time series analysis: time-step features and lag
features \parencite{haben2023time}. Time-step features indicate the interval between
two timestamps, while lag features represent the difference in a feature metric
compared to the previous timestamp (Table \ref{tab:time_series_extended}).


\begin{table}[h]
    \centering
    \begin{tabular}{|c|c|c|c|}
        \hline
        \textbf{Timestamp} & \textbf{Feature\_1} & \textbf{Time-step} & \textbf{Lag} \\
        \hline
        2000-04-01         & 100                 & 0                  & NaN          \\
        \hline
        2000-04-02         & 150                 & 1                  & 100          \\
        \hline
        2000-04-03         & 170                 & 2                  & 150          \\
        \hline
    \end{tabular}
    \caption{A simple time series extended with the time-step and lag feature.}
    \label{tab:time_series_extended}
\end{table}

Time-step features allow for modeling time dependence. For example, if the time-step
feature counts the days within the month, this can help identifying seasonal
dependencies of \textit{Feature\_1} within the month (Table \ref{tab:time_series_extended}).
On the other hand, lag features help identify serial dependence, meaning the correlation of
current feature values with past feature values. Therefore, serial dependence can test the
hypothesis whether past feature values influence current feature values.

\textcite{box2015time} defined the term autocorrelation as a measure of correlation
between the values of a time series at different time lags. A positive autocorrelation
coefficient at a certain lag indicates that high values tend to follow high values and
low values tend to follow low values, while a negative coefficient suggests that high
values follow low values and vice versa \parencite[ch. 2]{box2015time}.

\subsection{Trend}

Another commonly used term in time series forecasting is the ``trend''. A
trend describes a long-term change in the mean of the time series \parencite[ch. 5]{haben2023time},
as shown in Figure \ref{fig:ts_trend}.

\begin{figure}[h]
    \centering
    \includegraphics[width=0.75\linewidth]{img/Time Series Trend.png}
    \caption{The time series of the temperature anomaly shows a clear upwards trend
        \parencite[p. 311]{mudelsee2019trend}.}
    \label{fig:ts_trend}
\end{figure}

The trend is the slowest moving part of a time series, but it is essential
to understanding movement patterns that influence the time series. A trend can
be calculated in different ways. One common approach is the \ac{SMA}, which
calculates the mean of values within a sliding window \parencite{klinker2011exponential}.
The larger the width of the sliding window, the slower the moving average adapts
to (temporal) changes within the time series. Trends can be linear or quadratic
for instance. A more sophisticated moving average is the \ac{EMA} which places
greater significance on the most recent data points \parencite{hansun2013new}.
Unlike \ac{SMA}, which assigns equal weight to all observations in the period,
\ac{EMA} applies a weighting factor that decreases exponentially
\parencite{klinker2011exponential}. This makes \ac{EMA} more responsive to
recent price changes, which is particularly useful in financial markets for
analyzing trends and making trading decisions \parencite{dzikevivcius2010ema}.

Besides computing the trend through a moving average, we can also try to identify
it using a machine learning model, such as a linear regression. Using this trend, we
can then forecast future feature values by extending the learned function
representing the underlying trend within the data over future timestamps
\parencite{prophetpaper}.

\subsection{Seasonality}
\label{subsec:seasonality}

A recurring, periodic trend is often referred to as seasonality. A seasonality
is said to occur when the mean of the series changes repeatedly in the same
point in time within a given time range, e.g., year or month \parencite{haben2023time},
as shown in Figure \ref{fig:ts_seasonality}.
Seasonality can be caused by natural cycles or societal behaviors
related to dates or times. Examples of seasonality include temperatures peaking
in summer and toy sales peaking around Christmas  \parencite{haben2023time}.

\begin{figure}[h]
    \centering
    \includegraphics[width=0.75\linewidth]{img/Seasonality Time Series.png}
    \caption{A time series showing a clear example of seasonality \parencite{hyndman2011cyclic}.}
    \label{fig:ts_seasonality}
\end{figure}

To identify seasonality, seasonal plots and indicators can be used. A seasonal plot
shows segments of a time series against a consistent period,
representing the specific ``season'' you want to analyze. For instance, the values of a
feature can be plotted with the weekday on the x-axis if a weekly seasonality is
expected. Seasonal indicators can help a model distinguish means within a seasonal
period. For instance, if you expect weekly seasonality, the seasonal indicators
could be the weekdays in a one-hot-encoded format, as shown in Table \ref{tab:time_series_simple}.

\begin{table}[h]
    \centering
    \begin{tabular}{|c|c|c|c|c|c|c|}
        \hline
        \textbf{Timestamp} & \textbf{Feature\_1} & \textbf{Monday} & \textbf{Tuesday} & \textbf{Wednesday} & \textbf{Thursday} & \textbf{Friday} \\
        \hline
        2000-04-01         & 100                 & 1               & 0                & 0                  & 0                 & 0               \\
        \hline
        2000-04-02         & 150                 & 0               & 1                & 0                  & 0                 & 0               \\
        \hline
        2000-04-03         & 170                 & 0               & 0                & 1                  & 0                 & 0               \\
        \hline
        2000-04-04         & 180                 & 0               & 0                & 0                  & 1                 & 0               \\
        \hline
        2000-04-05         & 90                  & 0               & 0                & 0                  & 0                 & 1               \\
        \hline
    \end{tabular}
    \caption{A simple time series with seasonal indicators for the weekdays.}
    \label{tab:time_series_simple}
\end{table}

Note, that we can also model seasonality using sine and cosine curves to capture
multiple seasonalities within a time series. The technique which leverages these
curves to transform a time series into its constituent frequencies is called
Fourier Transformation \parencite[ch. 1]{bloomfield2004fourier}. By decomposing
a time series into a sum of sine and cosine functions, it enables the identification
and characterization of various cyclical patterns within the data. \ac{FFT} is
an efficient algorithm that computes the Fourier Transformation, making
it feasible to apply this method to large datasets in real-time applications
\parencite[ch. 5]{bloomfield2004fourier}. This approach allows for a more nuanced
understanding and forecasting of time series data by isolating and analyzing
recurring patterns across different temporal scales.

\subsubsection*{Cyclic Time Series}
Another notable mention is cyclic time series, a type of data sequence characterized
by periodic fluctuations that occur over irregular intervals, often spanning several
years. Unlike seasonal variations, which follow a regular, predictable pattern within
a year, cyclic patterns are influenced by broader economic, social, or environmental
factors and do not have a fixed period \parencite[p. 4102]{gharehbaghi2017deep}.
Possible reasons for those cycles are long-term influences such as business cycles,
economic expansions, and recessions.

\section{Forecasting}
\label{sec:Forecasting}
Time series forecasting is a statistical method used to predict future values based
on previously observed data points \parencite[ch. 5]{box2015time}. While there are
methods such as \ac{ARIMA} or exponential smoothing \parencite[ch. 5]{box2015time},
machine learning algorithms can also be employed to analyze the dependencies and
relationships between different time points. This allows us to learn from historical
data in order to predict future values. Generally, these machine learning algorithms
optimize their model parameters to minimize
forecasting errors \parencite[ch. 5]{box2015time}. Which parameters they optimize is
dependent on the underlying model. Popular models will be covered in Chapter
\ref{sec:forecasting_models}.

Typical examples for those forecasting erros are: the \ac{MAE}, which calculates
the average absolute difference between the predicted values and the actual values;
the \ac{MSE}, which computes the average of the squared differences between the
predicted and actual values (squaring the errors penalizes larger errors more
significantly, making MSE sensitive to outliers); the \ac{RMSE} which is the
square root of the \ac{MSE}, providing an error measure in the same units as
the original data and the \ac{MAPE} which expresses the error as a percentage
of the actual values, which can be useful for comparing forecast accuracy across
different datasets or scales.

For regression models, a common metric is the R-squared (R2) score, which
indicates how well the model's predictions approximate the variance in the data compared
to a simple mean prediction. The R-squared score ranges from 0 to 1, where 1
indicates a perfect fit. Additionally, the R-squared score can be adjusted to
account for the number of predictors in the model, providing a more realistic
evaluation of model fit. This adjusted metric is then referred to as the adjusted
R-squared score.

\section{Feature Engineering}
Feature engineering is crucial in time series forecasting, involving techniques
such as partial autocorrelation and autocorrelation analysis to identify relevant
lags. Autocorrelation measures how current values in the series relate to past values,
while partial autocorrelation isolates the relationship of specific lags, which helps
in the selection of relevant features for the model \parencite{box2015time}.

Further, one can also create new features such as lagged variables, which are created
by shifting the original timeseries data by one or more periods. The lags can be created
heuristically e.g. 1, 2 and 12 months. Consequently, \ac{PACF} can be used to identify
significant lags that should be included as features \parencite[ch. 6]{hyndman2018forecasting}.
These lagged variables can be combined with rolling windows which calculate statistics
(e.g. mean, standard deviation, sum, etc.) over a specific timeframe. After creating
lagged variables, the rolling windows can be used to calculate rolling statistics on
the features and capture more complex patterns \parencite[ch. 6]{hyndman2018forecasting}.

Additionally, it may be useful to include binary or numerical features representing
the occurrence of significant external events, e.g., holidays or economic shifts
\parencite[ch. 3]{hyndman2018forecasting}. Additionally, integrating real-time data as
features relevant to the modeling domain can significantly enhance the model's predictive
accuracy and responsiveness to current trends \parencite{2d9a26d795dd48a5a890e34035ac1882}.

\section{Forecasting Models}
\label{sec:forecasting_models}
Regarding time series forecasting models, one can distinguish between global and local
models which differ fundamentally in their scope and application \parencite{montero2021principles}.
Local models concentrate on individual time series, developing distinct models
tailored to each specific dataset. This
approach allows for highly customized forecasts tailored to the specific patterns and
characteristics of each series, often yielding high accuracy for single series predictions.
However, local models can be resource-intensive, demanding substantial computational power and
specialized expertise to develop and maintain multiple individualized models. In contrast,
global models aggregate
multiple time series into a single model, leveraging shared patterns and trends across datasets.
This approach can be more efficient, as it reduces the need for multiple models and can
improve scalability. Global models are particularly effective when individual series
exhibit similar behaviors, allowing the model to learn from a broader context. However,
they may not capture the unique nuances of each series as effectively as local models.
Ultimately, the choice between local and global time series forecasting depends on the
specific requirements of the application, including the number of series to forecast,
computational resources, and the need for individualized predictions
\parencite{montero2021principles}. Examples of local and global models can be found in
Table \ref{tab:timeseries_models}.

\begin{table}[h]
    \centering
    \begin{tabular}{ll}
        \toprule
        \textbf{Model}                                          & \textbf{Local/Global} \\
        \midrule
        Linear Regression                                       & Local                 \\
        \midrule
        \ac{ARIMA}                                              & Local                 \\
        \midrule
        \ac{ETS}                                                & Local                 \\
        \midrule
        Trees (Decision Tree, Random Forest, Gradient Boosting) & Local/Global          \\
        \midrule
        \ac{RNNs}                                               & Local/Global          \\
        \midrule
        \ac{LSTM}                                               & Local/Global          \\
        \midrule
        Transformer                                             & Local/Global          \\

        \bottomrule
    \end{tabular}
    \caption{Time series forecasting models \parencite{montero2021principles}.}
    \label{tab:timeseries_models}
\end{table}

There are various algorithms with which a time series forecasting problem can be
solved \parencite{salinas2020deepar}. Regression algorithms, such as linear and
polynomial regression, are foundational techniques that model the relationship
between independent variables and the target variable, providing interpretable
and straightforward forecasts. Tree-based methods, including decision trees,
random forests, and gradient boosting models, are powerful for handling complex,
non-linear relationships and interactions among features, often delivering robust
performance with less need for extensive data preprocessing. \ac{RNNs} excel in
time series forecasting due to their ability to maintain memory of previous inputs,
making them suitable for capturing temporal dependencies. \ac{LSTM} networks, an
advanced form of \ac{RNNs}, further enhance forecasting capabilities by addressing
the vanishing gradient problem, enabling the model to learn and retain long-term
dependencies more effectively. Each of these algorithms can be tailored to specific
forecasting challenges, with selection depending on factors such as data characteristics,
desired interpretability, and computational resources.

Additionally, deep learning frameworks have significantly advanced time series
forecasting by offering sophisticated methods for modeling complex patterns and
dependencies. Frameworks like DeepAR \parencite{salinas2020deepar} and DeepVAR
\parencite{cheng2020deepvar} are notable examples. DeepAR utilizes auto-regressive
\ac{RNNs} to provide probabilistic forecasts, excelling in
scenarios with large datasets by learning from the historical data of all time
series in the dataset, thereby capturing shared patterns and improving forecast
accuracy. DeepVAR extends this approach by incorporating multivariate time series
data, allowing the model to understand and leverage the interdependencies between
multiple series for more nuanced predictions. Recently, zero-shot time series
forecasting with \ac{LLM}s such as Chronos \parencite{ansari2024chronos} has
emerged as a cutting-edge approach. Chronos leverages pre-trained \ac{LLM}s to
forecast time series without the need for task-specific training, utilizing the
model's ability to generalize from vast amounts of data. This approach can
significantly reduce the time and computational resources required for model
development, making it an attractive option for rapid deployment in diverse
forecasting scenarios.

\subsection*{Hybrid Models}
Hybrid models are sophisticated models that integrate at least two distinct modeling approaches.
By combining the strengths of each model, hybrid models effectively mitigate their individual
weaknesses. This technique involves training the first model on the original time series data
and then training the second model on the residuals (errors) of the first model. The final
prediction is obtained by adding the outputs of both models. A crucial aspect is that each
model focuses on different features. For instance, if the first model is designed to predict
the trend, the second model does not need to include trend-related features. Some hybrid
models consist of several components; for example, a model could include separate components
for predicting the trend, yearly seasonality, weekly seasonality, and cycles
\parencite{zunic2020application}. Ensemble models that combine the results from
multiple underlying models in this manner are often referred to as additive models.

A popular example of such an additive model is Facebook Prophet by \textcite{prophetpaper}.
The ensemble comprises multiple models, each specializing in predicting different aspects of the data.
The first model predicts the trend, which could be piecewise linear (allowing for multiple
changepoints, in which the growth rate can change) or logistic. The second model accounts
for the seasonality component (yearly, weekly or daily) using the Fourier series (see
Chapter \ref{subsec:seasonality}). Thirdly, one of the models accounts for holiday effects,
as holidays can cause significant deviations from the usual patterns. The user can input a
list of holidays and the model will include their effects. Moreover, Prophet can include
additional regressor effects if the user provides external factors that are believed to
affect the time series. This results in the equation:

$$y(t) = g(t)+s(t)+h(t)+\epsilon(t)$$

where $y(t)$ is the observed value at time $t$, $g(t)$ is the trend component, $s(t)$ is
the seasonality component, $h(t)$ is the holiday effect and $\epsilon(t)$ is the error term,
capturing noise and other unexplained variability \parencite{prophetpaper}.


\chapter{Application}
\label{ch:application}
In this chapter, we describe how we leveraged the time series forecasting techniques
described in the previous chapter to predict earthquakes. We wrap our forecasting
model in a web application and enrich the dashboard with an AI assistant
to enable human-like interactions with the forecasting results.

Together, the dashboard (Appendix Figure \ref{fig:dashboard}) and
AI assistant (Appendix Figure \ref{fig:copilot}) aim at improving our ability to forecast
earthquakes and prepare for their consequences. This chapter explores the functionalities,
underlying technologies, and real-world applications of these tools, highlighting their
potential to revolutionize earthquake forecasting and enhance public safety.

Within the web application there are two main workflows, the first one being an overview and
analysis of recent earthquakes and the second being a conversation with a copilot to
support insights and decision making. Screenshots of the application can be found in
Appendix \ref{ch:web-application-design}.

\section{Modeling Problem}

As earthquakes are not equally distributed over the surface of the earth,
we decided to forecast earthquakes for the 25 most endangered regions on
the planet. This facilitates the forecasting process while only marginally
reducing our use case. To fulfill our goal of warning people about potential
dangers we effectively need to forecast the magnitude, depth and location of
an upcoming earthquake. A reliable prediction of these three features would
enable users to evaluate their situation and take preliminary action if needed.

We have chosen to forecast only the magnitude and depth of earthquakes and
to use historical median values for the location. This decision is based
on the fact that most earthquakes occur at the edges of tectonic plates,
resulting in similar location values for earthquakes within a given region.
By removing the location values as a forecasting target, the modeling problem
becomes more computationally efficient.

\subsection{Data}
To forecast, we utilized data from the \ac{USGS} Earthquake Catalog, accessed via
the \ac{FDSN} Event Web Service. This service provides comprehensive information
on seismic events using various parameters such as time, location, depth, and magnitude.
We queried the catalog for earthquakes from the past 100 years. Due to
limited data prior 1970, we opted to use the data from the past 30 years
for training (Figure \ref{fig:world-map}). Given our focus on building an end-to-end machine learning
application, we opted to use this free API, which allows us to forecast
future earthquakes based on live data. This approach closely resembles a
production system, enhancing the realism and practicality of our forecasting
model. The dataset, available in formats such as GeoJSON, XML, and CSV, was
instrumental in analyzing seismic activity trends and developing the forecasting model.


\begin{figure}[hbtp]
  \centering
  \includegraphics[scale=0.13]{img/world-earthquakes-top-25-regions-past-30-years.png}
  \captionsetup{format=hang}
  \caption{\label{fig:world-map}Top 25 regions with the highest number of earthquakes since 1974.}
\end{figure}

The dataset contained detailed information about the time, location (latitude and longitude),
depth, and magnitude of an occurring earthquake. Additionally, there are other features
available on the API. A detailed list can be found on the \ac{USGS} website
\footnote{\url{https://earthquake.usgs.gov/earthquakes/feed/v1.0/csv.php}}.

\subsubsection{Preprocessing}

Effective preprocessing is critical for handling the unique characteristics of
time series data, particularly in earthquake forecasting. In our study, we began
by addressing the unevenly spaced timestamps in the dataset. To standardize the
intervals, we opted for daily aggregation. While we also experimented with hourly
aggregation, it resulted in prohibitively long model training times - ranging
from 12 hours to a full day - making daily aggregation the more practical choice.

Our analysis revealed a strong time dependency for most values, extending up to
40 lags and potentially more in certain regions (Figure \ref{fig:pacf-sample};
\ref{fig:mag-acf-sample}; \ref{fig:depth-acf-sample}). This suggests that extensive
historical data would be beneficial for predictions. However, incorporating
such long historical windows significantly increases dimensionality, making
the model complex and computationally expensive. We question the practicality
of using nearly every recorded value in a region for forecasting future earthquake
occurrences, especially given that some regions have lags spanning multiple months or even years.
To address the dimensionality challenge, we capped the historical lags and the
exponential moving average at 7 days into the past. This compromise ensured that
we incorporated sufficient historical information while keeping the model
computationally feasible. Balancing the need for historical data without
overwhelming the model with too much information was crucial for our approach.

Our analysis of partial autocorrelation and autocorrelation functions
(Figure \ref{fig:pacf-sample};
\ref{fig:mag-acf-sample}; \ref{fig:depth-acf-sample}) revealed
that only the first few lags were significantly correlated in some regions.
This finding indicated that adding more lags beyond this point would not
improve predictive accuracy but would instead introduce unnecessary complexity.
Thus, capping historical lags at 7 days further validated our approach, focusing
model training on the most relevant temporal dependencies for effective earthquake
prediction in those specific regions.

\begin{figure}[hbtp]
  \centering
  \includegraphics[scale=0.2]{img/pacf-sample.png}
  \captionsetup{format=hang}
  \caption{\label{fig:pacf-sample}Magnitude and depth partial autocorrelation of Alaska.
    The full details can be found in Appendix Figure \ref{fig:mag-pacf} (magnitude)
    and Figure \ref{fig:depth-pacf} (depth).}
\end{figure}

\begin{figure}[hbtp]
  \centering
  \includegraphics[scale=0.2]{img/magnitude-acf-sample.png}
  \captionsetup{format=hang}
  \caption{\label{fig:mag-acf-sample}Magnitude autocorrelation of Alaska and Oregon.
    The full details can be found in Appendix Figure \ref{fig:mag-acf}.}
\end{figure}

\begin{figure}[hbtp]
  \centering
  \includegraphics[scale=0.2]{img/depth-acf-sample.png}
  \captionsetup{format=hang}
  \caption{\label{fig:depth-acf-sample}Depth autocorrelation of Alaska and Indonesia.
    The full details can be found in Appendix Figure \ref{fig:depth-acf}.}
\end{figure}

Given that we are conducting global time series forecasting, these measures are
sufficient and appropriate. By capping the historical lags at 7 days, we successfully
balanced the need for historical information with the necessity of maintaining a
computationally feasible model. This strategy, backed by our autocorrelation analysis,
ensures that our model remains both efficient and accurate in predicting earthquake
occurrences on a global scale.

\begin{figure}[hbtp]
  \centering
  \includegraphics[scale=0.2]{img/periodogram-sample.png}
  \captionsetup{format=hang}
  \caption{\label{fig:periodogram-sample}Magnitude and depth periodogram of Idaho.
    The full details can be found in Appendix Figure \ref{fig:mag-periodogram} (magnitude)
    and Figure \ref{fig:depth-periodogram}.}
\end{figure}

Furthermore, our analysis of depth and magnitude periodograms (Figure \ref{fig:periodogram-sample})
indicated that earthquakes
exhibit a high degree of randomness, with some seasonality detected in only a minority
of regions. Due to this predominant randomness, using \ac{FFT}
to model seasonality proved impractical. Consequently, our preprocessing did not
rely on \ac{FFT}-based seasonal adjustments.

For model training, we implemented an 80-20 train-test split for each region. This
split was time-based, ensuring that past data was used to predict future values.
This method aligns with the temporal nature of the data and the forecasting goal,
allowing the model to learn from historical trends and apply this knowledge to
future predictions.

\subsection{Model}

To forecast earthquakes for 25 regions accurately, we employ a global forecasting model.
Among the various approaches available, tree-based models and \ac{RNNs}, specifically
\ac{LSTM} networks, are considered due to their ability to handle time-series data
effectively. For this particular application, we use CatBoost regression
\parencite{prokhorenkova2018catboost}, an efficient gradient boosting framework.

We chose CatBoost regression for forecasting earthquakes across 25 regions due to
its handling of categorical features, which simplifies preprocessing and ensures
robust feature encoding. CatBoost also offers better interpretability, allowing us
to understand the impact of different variables on predictions. Its training
efficiency and built-in mechanisms to prevent overfitting make it a practical
choice, particularly with the volatile nature of earthquake data. Additionally,
CatBoost can capture complex feature interactions without extensive feature
engineering, streamlining the modeling process and enhancing overall accuracy.

While tree-based models often struggle with forecasting trends, CatBoost regression
is particularly well-suited for our task for several reasons. The magnitude of
earthquakes is capped within a range of [-1, 10], and their depth falls within
  [-100, 1000] meters \parencite{earthquake-data}. In regression trees, predictions
are made by traversing a series of if-else conditions to reach a leaf node, where
the average of the values in that leaf are used for the prediction. Due to the
nature of these trees, it is not possible to predict values outside of those
observed in the training set. This makes CatBoost, with its advanced handling
of categorical features and robust performance on unseen data, an excellent
choice for this type of regression problem.

CatBoost transforms categorical features into numerical ones by employing
several methods, including random permutation of input objects and label
value conversion to integers, tailored to the type of machine learning problem
(regression, classification, or multi-classification). It calculates the
transformation using various statistics, such as counting occurrences within
specific buckets or calculating mean target values. The primary formula for
this transformation is:

\[\text{ctr} = \frac{\text{countInClass} + \text{prior}}{\text{totalCount} + 1}\]

where \textit{countInClass} is the relevant label count, \textit{prior} is a
predefined constant, and \textit{totalCount} is the cumulative count of objects
with the same categorical value. Additionally, CatBoost supports one-hot encoding
for categorical features with limited unique values, controlled by the
\textit{one\_hot\_max\_size} parameter \parencite{catboost-encoding}.

To forecast earthquakes, we predict magnitudes and depths, while for location
(latitude and longitude), we compute the historic median to reduce model
complexity and enhance stability. The forecasting process is performed
recursively or iteratively: initially, we forecast the data for one day,
using the predicted values to create features for the subsequent day.
This process is repeated, building on each previous prediction, until
forecasts for three days are generated. This recursive approach allows
the model to adapt dynamically to the changing conditions and trends
observed in the time series data, providing more accurate and reliable forecasts.

\subsubsection{Training}
The model training process for earthquake forecasting requires a structured
approach, beginning with a careful feature selection. The chosen features
include temporal attributes such as the day, day of the week, and day of
the year, which capture the time-dependent aspects of earthquake occurrences.
Additionally, exponential moving averages for both magnitude and depth are
included to reflect the trends over recent periods. The model also incorporates
lagged values for magnitude and depth over a specified range, allowing it to
consider past events' influence on current conditions. The inclusion of the
categorical feature region ensures that the model can account for regional
differences in earthquake behavior.

The chosen model for this task is the CatBoost Regressor, selected for its
ability to handle categorical features natively and its strong performance
with complex datasets. The model is configured with early stopping rounds to
mitigate overfitting and is optimized using the MultiRMSE loss function, which
targets minimizing errors in both magnitude and depth predictions. The training
dataset comprises the selected features and the categorical region feature,
while the target variables are the earthquake magnitude and depth.

To enhance the model's performance, hyperparameter tuning is performed using
grid search, as shown in Table \ref{table:hyperparameters}. This involves
testing various combinations of hyperparameters
such as learning rate, model depth, and L2 leaf regularization. The grid
search evaluates these combinations to identify the optimal parameters that
yield the best predictive performance. This systematic approach to feature
selection, data preparation, and hyperparameter tuning aims to develop a
robust and accurate model capable of forecasting earthquake magnitude and
depth across different regions.

\begin{table}[h!]
  \centering
  \begin{tabular}{ll}
    \toprule
    \textbf{Hyperparameter} & \textbf{Values}          \\ \midrule
    Learning Rate           & 0.01, 0.03, \textbf{0.1} \\ \midrule
    Depth                   & 4, 6, \textbf{10}        \\ \midrule
    L2 Leaf Regularization  & 1, 3, \textbf{5}, 7, 9   \\ \bottomrule
  \end{tabular}
  \caption{Hyperparameter values explored during grid search. Best hyperparameters highlighted in bold.}
  \label{table:hyperparameters}
\end{table}

\subsubsection{Evaluation}
After training the CatBoost Regressor model for earthquake forecasting,
it is crucial to evaluate its performance using several metrics and visualization techniques.
The performance of the time series forecasting model is assessed using a variety of
metrics that provide a comprehensive view of its accuracy and reliability. The selected
metrics are the \ac{MAE}, the \ac{RMSE}, the R2-Score and the adjusted R2 Score
(Table \ref{table:model-metrics}).
These metrics collectively offer insights into the error magnitude and the goodness-of-fit
of the model. The rationale behind these metrics is explained in \ref{sec:Forecasting}.

\begin{table}[H]
  \centering
  \begin{tabular}{ll}
    \toprule
    \textbf{Metric} & \textbf{Score} \\ \midrule
    MAE             & 2.9344         \\ \midrule
    RMSE            & 10.3577        \\ \midrule
    R2              & 0.9495         \\ \midrule
    Adjusted R2     & 0.9495         \\ \bottomrule
  \end{tabular}
  \caption{Model metrics.}
  \label{table:model-metrics}
\end{table}

For our model, the MAE is 2.9344 (Figure \ref{table:model-metrics}), indicating
that on average, the predictions are approximately 2.9344 units away from the
actual values. Lower MAE values suggest better predictive accuracy, and the
obtained MAE value demonstrates that the model has a reasonably low level of error.

The RMSE for our model is 10.3577 (Figure \ref{table:model-metrics}),
which reflects the model's error magnitude.
While RMSE values are generally higher than MAE due to the squaring of errors,
they provide an important perspective on how significant the larger errors are
in the context of the model's predictions. A lower RMSE indicates a better fit,
and our RMSE value suggests a reasonable performance with some sensitivity to larger errors.

For our model, the R2 is 0.9495 (Figure \ref{table:model-metrics}), signifying
that approximately 94.95\% of the variance in the data is explained by the model.
This high R2 value suggests that the model fits the data very well and is capable
of capturing the underlying patterns in the time series.

The Adjusted R2 for our model is 0.9495 (Figure \ref{table:model-metrics}),
which is very close to the R2 value. This indicates that the inclusion of
predictors in the model is justified and that the model does not overfit the data.
The slight difference between R2 and Adjusted R2 confirms that the model maintains
a high level of explanatory power while accounting for the number of predictors.

The evaluation of the model using these metrics provides a well-rounded
understanding of its performance. The low MAE and RMSE values indicate
that the model has a low level of error, while the high R2 and Adjusted
R2 values demonstrate a strong fit to the data. Together, these metrics
suggest that the model is both accurate and reliable for forecasting the
time series data, making it a valuable tool for predicting future values.

In Appendix Figure \ref{fig:mag-forecast-full}, the magnitude forecasts demonstrate
varying degrees of accuracy across the different regions. For regions
like Alaska and Nevada, the model's predictions closely follow the
actual magnitudes, indicating a high level of accuracy. However, in
regions such as Indonesia and Chile, there are noticeable deviations
between the forecasted and actual values, suggesting areas where the
model could be improved. Overall, while the model captures the general
trends and fluctuations in earthquake magnitudes reasonably well, the
precision varies significantly by region. A sample of magnitude forecasts
is illustrated in Figure \ref{fig:mag-forecast}.

\begin{figure}[hbtp]
  \centering
  \includegraphics[scale=0.2]{img/magnitude-forecast-sample.png}
  \captionsetup{format=hang}
  \caption{\label{fig:mag-forecast}Magnitude forecast sample of Alaska (solid performance),
    Philippines (good performance), and Wyoming (suboptimal performance).}
\end{figure}

Appendix Figure \ref{fig:depth-forecast-full} focuses on the depth forecasts.
Similar to the magnitude forecasts, the depth predictions show
regional variability in accuracy. In regions such as Alaska,
California, and Nevada, the forecasted depths align closely
with the actual depths, indicating strong predictive performance.
Conversely, in regions like Indonesia and Papua New Guinea, the
forecasted depths show more significant discrepancies from the
actual values, highlighting potential areas for model enhancement.
A sample of depth forecasts is illustrated in Figure \ref{fig:depth-forecast}.

\begin{figure}[hbtp]
  \centering
  \includegraphics[scale=0.2]{img/depth-forecast-sample.png}
  \captionsetup{format=hang}
  \caption{\label{fig:depth-forecast}Depth forecast sample of Oregon (good performance),
    Utah (suboptimal performance), and Washington (solid performance).}
\end{figure}

Combining the insights from both sets of forecasts, it is evident
that the model's performance is influenced by regional characteristics.
The regions with higher accuracy in both magnitude and depth forecasts,
such as Alaska and Nevada, suggest that the model effectively captures
the underlying patterns in these areas. In contrast, regions with lower
accuracy, such as Indonesia and Chile, may require additional data or
model adjustments to improve predictive performance. Overall, the
model demonstrates a better capability for forecasting future depths than magnitudes.

While the model's evaluation metrics, including \ac{MAE}, \ac{RMSE}, R2, and Adjusted R2,
indicate excellent performance in terms of accuracy and goodness-of-fit,
it is important to acknowledge a significant limitation: the inherent
unpredictability of earthquakes. Earthquakes are fundamentally random
events, much like predicting stock market values, making them
exceptionally challenging to forecast with high precision. Despite
the model's robust statistical indicators, the chaotic and complex
nature of seismic activities means that accurate prediction remains
elusive. Thus, while the model shows promising results according to
the chosen metrics, its practical applicability in earthquake
forecasting is constrained by the unpredictable and stochastic
nature of the phenomenon itself.

\begin{figure}[hbtp]
  \centering
  \includegraphics[scale=0.18]{img/model-feature-importance.png}
  \captionsetup{format=hang}
  \caption{\label{fig:feature-importance}Feature importance}
\end{figure}

The feature importance plot displayed in Figure \ref{fig:feature-importance}
highlights the significance of various input features in a forecast model that
predicts the magnitude and depth of earthquakes. Notably, the feature
\textit{depth\_ewma} (exponential moving average of depth) stands out as
the most critical feature by a considerable margin, indicating its paramount
influence on the model's predictions. Following \textit{depth\_ewma},
features such as \textit{depth\_lag\_1}, \textit{depth\_lag\_2}, and
\textit{depth\_lag\_3} also demonstrate significant importance, suggesting
that recent depth values have a substantial impact on the forecast. In
contrast, features related to the earthquake magnitude, such as
\textit{mag\_ewma} (exponential moving average of magnitude) and various
magnitude lags, show much lower importance, indicating a lesser role in
the prediction model. Additionally, temporal features like \textit{dayofyear}
and \textit{dayofweek} have minimal impact. This analysis underscores that
the recent history of earthquake depth, especially its exponential moving
average, is the most influential factor in forecasting future earthquake
magnitudes and depths.

\section{LLM-powered AI Agent}
Advancements in AI have given rise to AI agents as powerful tools
for enhancing productivity and communication \parencite{xi2023rise}.
Equipped with \ac{LLM}s, these agents can perform a wide array of tasks,
such as information processing, creative text generation, language
translation, and providing insightful responses to inquiries.

Unlike conventional virtual assistants and chatbots, AI agents operate
based on learned patterns and data, enabling them to offer more nuanced
understanding and communication. For instance, an AI agent trained on
extensive datasets of text and code can retrieve pertinent articles,
distill key findings into summaries, or craft tailored content based
on the preferences of specific target audiences. The potential
applications of LLM-powered AI agents are vast and diverse, ranging
from automating repetitive tasks to supporting complex decision-making
processes and creative endeavors. However, it is crucial to recognize
that while AI agents excel in language manipulation and generation, they
lack true sentience or consciousness, necessitating responsible usage to
mitigate the risk of biased or misleading content generation.

\subsection{Instructions}

The AI agent uses a combination of tools to deliver accurate and timely
earthquake forecasts. The agent is built using the ReAct prompting
\parencite{yao2023reactsynergizingreasoningacting} method as shown in
Source Code \ref{lst:system-prompt}, which enhances its interaction
capabilities and allows it to provide expert-level responses regarding earthquakes.
An example of the LLM-powered AI agent in action can be found in Appendix Figure
\ref{fig:copilot}.

\begin{lstlisting}[caption={\texttt{system\_prompt.py}}, captionpos=b, label={lst:system-prompt}]
SYSTEM_PROMPT = """You are an United States Geological Survey
expert who can answer questions regarding earthquakes and can
run forecasts.

Before you use the Forecast Earthquakes tool, always check
which regions are available using Find Regions first.
Respond to the user as helpfully and accurately as possible.

You have access to the following tools:
{tools}

Please ALWAYS use the following JSON format:
{{
  "thought": "Explain your thought. Consider previous and
    subsequent steps",
  "tool": "The tool to use. Must be on of {tool_names}",
  "tool_input": "Valid keyword arguments (e.g. {{"key": value}})",
}}

Observation: tool result
... (this Thought/Tool/Tool input/Observation can repeat N times)

When you know the answer, you MUST use the following JSON format:
{{
  "thought": "Explain the reason of your final answer
    when you know what to respond",
  "tool": "Final Answer",
  "tool_input": "Valid keyword arguments (e.g. {{"key": value}})",
}}"""
\end{lstlisting}

ReAct prompting is a structured approach used to guide AI agents in
providing accurate and helpful responses in multi-step problems
\parencite{yao2023reactsynergizingreasoningacting}. The agent is
programmed with a specific system prompt that positions it as a
United States Geological Survey (USGS) expert. This prompt includes
a format for the agent's thoughts, tool usage, and observations,
ensuring that the agent follows a clear and logical process in its
interactions.

Even though we define a specific output format, the LLM still predicts
the most likely token that should come next, which can lead to errors
in the output format, such as responding with text instead of JSON.
This process can be compared to a trial-and-error learning process.
We then enter a loop of refinement. If the LLM's output can't be
parsed into the correct format, it's reminded of the requirement
and prompted to adjust. Conversely, a successful parse leads to the
LLM's output being used to call a relevant tool. If the tool call
fails, the LLM is asked to refine it further. This cycle continues
until a satisfactory answer is produced, either through a parsable
output or the LLM exhausting its attempts. This iterative process
ensures continuous feedback and correction, ultimately aiming to
achieve the desired outcome: a correct response in the specified format.

Explainable AI agents play a crucial role in fostering trust and
transparency with users by allowing them to understand the
decision-making processes of \ac{LLM}s. Through the use of
techniques such as ReAct prompting \parencite{yao2023reactsynergizingreasoningacting},
we can visibly demonstrate what the LLM is doing behind the scenes. This
involves showing the sequence of actions taken by the model, including
its thoughts, the tools employed, the input provided to these tools,
the tools' responses, and the model's subsequent thoughts leading to
the final answer. By offering this level of insight, we can build
greater confidence in the LLM's outputs and facilitate a more
informed interaction with the technology. An example is shown in Appendix Figure
\ref{fig:explainable-ai}.

\subsection{Tools}

The AI agent designed for earthquake forecasting integrates several tools
to provide accurate and comprehensive information. These tools, leveraging
forecasting models and data processing techniques, enhance the agent's
capability to predict and analyze seismic events. Here’s a detailed
look at the tools used by the AI agent:

\begin{enumerate}
  \item \textbf{Current Date:} This tool provides the current
        local date and time. It is essential for timestamping queries
        and ensuring that all data processing is aligned with the
        date the user asked about.
  \item \textbf{Query Earthquakes:} This tool searches for
        recent earthquakes based on various parameters such as start
        time, end time, depth, magnitude, and alert level. It utilizes
        the US Geological Survey (USGS) API to retrieve data in the
        geojson format.
  \item \textbf{Count Earthquakes:} This tool counts and aggregates
        recent earthquake events based on specified criteria. It uses the
        same parameters as the Query Earthquakes tool to filter the events.
  \item \textbf{Find Regions:} This tool retrieves the available
        regions for which earthquake forecasts can be made. It ensures
        that the LLM is aware of the regions supported by the forecasting
        model before making predictions.
  \item \textbf{Forecast Earthquakes:} This tool predicts future
        earthquakes in a specified region. It processes the forecast data
        to return predictions including the date, magnitude, and depth of
        potential earthquakes.
\end{enumerate}

\section{Deployment and Monitoring}
The first step in deploying our ML application involved containerizing
the model using Docker\footnote{\url{https://www.docker.com/}}. Docker
allowed us to package the model along with its dependencies into a
container, ensuring consistency across different deployment environments.
After saving the model, we created the Dockerfile and the Docker Image to
install dependencies and run the code.

For deploying the Flight Base App on Render.com, the team optimally
utilized the platform's versatile features. Render.com uses
Amazon Web Services (AWS) under the hood, which provided the
team with a solid infrastructure and worldwide availability through
the AWS data center in Frankfurt. The app deployment is automatic as
soon as changes are pushed to the team's GitHub repository and the
deployment branch is synchronized.

To handle the model's computational demands, especially for real-time
predictions, deploying the Docker container on a cluster with GPU support
would be ideal. Although we did not have access to a GPU-supported cluster
for this project, it is a practical approach when high performance is needed.
The deployment would typically be managed through a fully orchestrated workflow.
In practice, the training and deployment workflow are fully automated,
which can be implemented using Argo Workflows\footnote{\url{https://argoproj.github.io/workflows/}}.
However, this was out of scope for this project.

Additionally, when deploying models in practice, using a platform like
Kubeflow\footnote{\url{https://www.kubeflow.org/}} is beneficial. Kubeflow
leverages Kubernetes for machine learning and allows you to write templates
defining how to deploy models, specifying resources such as RAM, CPU, and GPU,
as well as managing canary rollouts and other deployment strategies.

The management of environment variables is handled directly through Render.com.
These are automatically inserted during the build of the Docker container and
are available to the application. Through this approach, the team ensured that
the entire deployment configuration remains clear and efficient, while also
ensuring a smooth live deployment of the application on Render.com. This
method minimizes the risk of misconfigurations and simplifies the management
of sensitive data such as API keys.

Furthermore, effective monitoring of the deployed model is crucial to ensure
that it continues to perform as expected. Our monitoring process includes
checking for model drift, both data drift and concept drift. Model drift occurs
when the statistical properties of the target variable (concept drift) or the
input features (data drift) change over time, leading to a decline in model
performance. An example for a concept drift in earthquake forecasting would
be a change in the relationship between seismic precursor signals
(like foreshocks, ground deformation, or gas emissions) and the occurrence
of an earthquake due to new geological activities or shifts in tectonic plate
interactions. Data drifts would include a change in sensor calibration, which
would lead to a shift in the data they collect. Additionally, the performance
metrics of the model on live data are monitored. Consequently, the model is
retrained if the performance metrics decline below a certain threshhold.

In summary, while this project did not employ advanced orchestration tools
or GPU-supported clusters due to constraints, these are critical considerations
for deploying robust and scalable machine learning models in a production environment.


% Fazit und Ausblick
% !TEX root =  master.tex
\chapter{Conclusion}\label{ch:conclusion}
To conclude, the project was successfully completed, and its goals were achieved.
We preprocessed the data and trained a robust model, which was then used to build a
comprehensive application. This application features a dashboard displaying recent
earthquakes and an interactive map showing future earthquakes for the next three days
worldwide. Given the extensive volume of the presented data, which might seem overwhelming,
we developed an AI agent to increase accessibility. This agent allows users to run
personalized forecasts tailored to their specific needs. Since most users are
interested in one or two regions rather than all 25 regions covered by the forecast,
this feature is particularly useful. Additionally, the AI agent acts as a \ac{USGS} expert,
providing users with in-depth analysis on potential risks and safety precautions for individual earthquakes.

Recognizing that many people are not tech-savvy and aiming to spread information
about earthquakes to a broad audience, we minimized technical barriers. By integrating
an interactive AI agent, we made it easier for users to access and understand crucial
earthquake information.

To enhance the predictive accuracy of earthquake forecasting models, several potential
improvements can be considered. Integrating live data such as magma flow, plate tectonics
movements, and real-time seismic activity could provide a more comprehensive and dynamic
understanding of the underlying processes that precede earthquakes. Additionally, exploring
advanced deep learning models, such as \ac{RNNs} and \ac{LSTM}s, which are well-suited for time series
data, might offer improved prediction capabilities by capturing complex temporal dependencies.
Furthermore, incorporating multi-source data fusion techniques, which combine various types of
geological and environmental data, could enhance the robustness of the model. Continuous
updates and retraining of the model with the latest data, along with cross-disciplinary
collaboration between seismologists and data scientists, would also contribute to refining
the forecasting accuracy.

However, it is important to acknowledge the inherent unpredictability of earthquakes.
Similar to predicting stock market values, earthquakes are fundamentally random and
complex events, making precise forecasting exceptionally difficult. Despite robust
statistical models, accurate predictions remain elusive. Nonetheless, these improvements
could collectively aid in better anticipating seismic events despite their chaotic nature.
%%%%%%%%%%%%%%%%%%%%%%%%%%%%%%%%%%%

%%%%%%%%%%%%%%%%%%%%%%%%%%%%%%%%%%%
% ANHÄNGE
%
% @stud: einzelne Anhänge bearbeiten und eigene Anhänge hier einfügen 
%        die nachfolgenden Zeilen deaktivieren, wenn keine Anhänge verwendet werden
%
\initializeAppendix
% !TEX root =  master.tex
\chapter{Web Application Design}\label{ch:web-application-design}

\begin{figure}[hbtp]
    \centering
    \includegraphics[scale=0.25]{img/dashboard.png}
    \captionsetup{format=hang}
    \caption{\label{fig:dashboard}Interactive dashboard displaying
        recent earthquakes, with a map featuring earthquake forecasts
        and a corresponding table detailing the forecasted values for
        magnitude and depth.}
\end{figure}

\begin{figure}[hbtp]
    \centering
    \includegraphics[scale=0.25]{img/copilot-forecast-example.png}
    \captionsetup{format=hang}
    \caption{\label{fig:copilot}Copilot interface showcasing human-AI
        collaboration, where an \ac{LLM}-powered AI agent assists by
        answering questions, running forecasts, and recommending
        precautionary actions.}
\end{figure}

\begin{figure}[hbtp]
    \centering
    \includegraphics[scale=0.35]{img/explainable-ai-agent.png}
    \captionsetup{format=hang}
    \caption{\label{fig:explainable-ai}Explainable AI in action:
        The \ac{LLM}-powered AI agent demonstrates its reasoning process as
        it runs a forecast for California, USA. By following the
        reasoning steps, the user can verify the accuracy and
        transparency of the \ac{LLM}'s operations, thereby building trust.}
\end{figure}


% % !TEX root =  master.tex
\chapter{Machine Learning Modeling}

\begin{figure}[hbtp]
    \centering
    \includegraphics[scale=0.15]{img/magnitude-pacf-per-region.png}
    \captionsetup{format=hang}
    \caption{\label{fig:mag-pacf}Magnitude partial autocorrelation per region.}
\end{figure}

\begin{figure}[hbtp]
    \centering
    \includegraphics[scale=0.15]{img/depth-pacf-per-region.png}
    \captionsetup{format=hang}
    \caption{\label{fig:depth-pacf}Depth partial autocorrelation per region.}
\end{figure}

\begin{figure}[hbtp]
    \centering
    \includegraphics[scale=0.15]{img/magnitude-acf-per-region.png}
    \captionsetup{format=hang}
    \caption{\label{fig:mag-acf}Magnitude autocorrelation per region.}
\end{figure}

\begin{figure}[hbtp]
    \centering
    \includegraphics[scale=0.15]{img/depth-acf-per-region.png}
    \captionsetup{format=hang}
    \caption{\label{fig:depth-acf}Depth autocorrelation per region.}
\end{figure}

\begin{figure}[hbtp]
    \centering
    \includegraphics[scale=0.15]{img/magnitude-periodogram.png}
    \captionsetup{format=hang}
    \caption{\label{fig:mag-periodogram}Magnitude periodogram:
        spectral density of the magnitude as a function of frequency.}
\end{figure}

\begin{figure}[hbtp]
    \centering
    \includegraphics[scale=0.15]{img/depth-periodogram.png}
    \captionsetup{format=hang}
    \caption{\label{fig:depth-periodogram}Depth periodogram:
        spectral density of the depth as a function of frequency.}
\end{figure}

\begin{figure}[hbtp]
    \centering
    \includegraphics[scale=0.15]{img/magnitude-forecast-test-set.png}
    \captionsetup{format=hang}
    \caption{\label{fig:mag-forecast-full}Magnitude forecast per region on test set.}
\end{figure}

\begin{figure}[hbtp]
    \centering
    \includegraphics[scale=0.15]{img/depth-forecast-test-set.png}
    \captionsetup{format=hang}
    \caption{\label{fig:depth-forecast-full}Depth forecast per region on test set.}
\end{figure}
%%%%%%%%%%%%%%%%%%%%%%%%%%%%%%%%%%%

\singlespacing

%\ihead{}
%\printbibliography[title=\Literaturverzeichnis] 
\printbibliography 
\cleardoublepage

%\initializeBibliography
%%%%%%%%%%%%%%%%%%%%%%%%%%%%%%%%%%%

%%%%%%%%%%%%%%%%%%%%%%%%%%%%%%%%%%%
% INDEX
% @stud: ggf. Index auskommentieren, wenn nicht benötigt
%
% \addcontentsline{toc}{chapter}{Index}
% \printindex

\end{document}
