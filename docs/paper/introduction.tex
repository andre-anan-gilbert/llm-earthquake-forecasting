% !TEX root =  master.tex

\chapter{Introduction}
Earthquakes are a common phenomenon in our world, with approximately 2,000 earthquakes
of magnitude 5 on the Richter scale occurring annually \parencite{welt_erdbeben}.
While these earthquakes are classified as moderate, they still have the potential
to be deadly and cause significant damage to infrastructure and communities. The
unpredictability of earthquakes poses a considerable challenge to disaster preparedness
and risk mitigation.

To mitigate the risks of injuries or death, our goal is to develop a forecasting model
capable of predicting when and where an earthquake might occur. Such a model would be
invaluable in providing early warnings, allowing for timely preparation and potentially
saving lives. By leveraging out-of-the-box forecasting models and adapting their methodology
to the use case of earthquake forecasting, we aim to build a machine learning model to
enhance the accuracy and reliability of earthquake forecasts.

To further expand our web platform, we plan to integrate an \ac{LLM}. This \ac{LLM}
will not only offer easy access to the calculated predictive information but will
also provide comprehensive insights into earthquakes. Our goal is to strengthen public
understanding of earthquakes and promote safety precautions. This additional resource
will serve as a tool, offering explanations on seismic activity, historical earthquake
data, and guidelines for preparedness. Thus, we aim not only to develop an approach
capable of predicting earthquakes but also to ensure long-term usage by the population
in endangered areas.

On one side, this page allows regions to stay ahead of potential earthquakes by offering
sufficient time for preparation or evacuation. Early warnings by our predictive model can
lead to better-organized emergency responses, reducing the impact of earthquakes on human
lives and property. By empowering communities with predictive data and educational resources,
we aim to enhance resilience against seismic hazards.

As a result, this paper is organized as follows: Chapter \ref{ch:forecasting} explains
forecasting and the theoretical background behind our approach in detail. Chapter
\ref{ch:application} describes the development and deployment of our model and web
application. Additionally, we evaluate the model's performance and feature importance.
Finally, we interpret the results and outline potential improvements and next steps in
Chapter \ref{ch:conclusion}.